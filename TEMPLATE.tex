% Options for packages loaded elsewhere
\PassOptionsToPackage{unicode}{hyperref}
\PassOptionsToPackage{hyphens}{url}
\PassOptionsToPackage{dvipsnames,svgnames,x11names}{xcolor}
%
\documentclass[
  11pt,
  letterpaper,
]{article}

\usepackage{amsmath,amssymb}
\usepackage{iftex}
\ifPDFTeX
  \usepackage[T1]{fontenc}
  \usepackage[utf8]{inputenc}
  \usepackage{textcomp} % provide euro and other symbols
\else % if luatex or xetex
  \usepackage{unicode-math}
  \defaultfontfeatures{Scale=MatchLowercase}
  \defaultfontfeatures[\rmfamily]{Ligatures=TeX,Scale=1}
\fi
\usepackage{lmodern}
\ifPDFTeX\else  
    % xetex/luatex font selection
\fi
% Use upquote if available, for straight quotes in verbatim environments
\IfFileExists{upquote.sty}{\usepackage{upquote}}{}
\IfFileExists{microtype.sty}{% use microtype if available
  \usepackage[]{microtype}
  \UseMicrotypeSet[protrusion]{basicmath} % disable protrusion for tt fonts
}{}
\makeatletter
\@ifundefined{KOMAClassName}{% if non-KOMA class
  \IfFileExists{parskip.sty}{%
    \usepackage{parskip}
  }{% else
    \setlength{\parindent}{0pt}
    \setlength{\parskip}{6pt plus 2pt minus 1pt}}
}{% if KOMA class
  \KOMAoptions{parskip=half}}
\makeatother
\usepackage{xcolor}
\usepackage[margin=1in]{geometry}
\setlength{\emergencystretch}{3em} % prevent overfull lines
\setcounter{secnumdepth}{5}
% Make \paragraph and \subparagraph free-standing
\makeatletter
\ifx\paragraph\undefined\else
  \let\oldparagraph\paragraph
  \renewcommand{\paragraph}{
    \@ifstar
      \xxxParagraphStar
      \xxxParagraphNoStar
  }
  \newcommand{\xxxParagraphStar}[1]{\oldparagraph*{#1}\mbox{}}
  \newcommand{\xxxParagraphNoStar}[1]{\oldparagraph{#1}\mbox{}}
\fi
\ifx\subparagraph\undefined\else
  \let\oldsubparagraph\subparagraph
  \renewcommand{\subparagraph}{
    \@ifstar
      \xxxSubParagraphStar
      \xxxSubParagraphNoStar
  }
  \newcommand{\xxxSubParagraphStar}[1]{\oldsubparagraph*{#1}\mbox{}}
  \newcommand{\xxxSubParagraphNoStar}[1]{\oldsubparagraph{#1}\mbox{}}
\fi
\makeatother

\usepackage{color}
\usepackage{fancyvrb}
\newcommand{\VerbBar}{|}
\newcommand{\VERB}{\Verb[commandchars=\\\{\}]}
\DefineVerbatimEnvironment{Highlighting}{Verbatim}{commandchars=\\\{\}}
% Add ',fontsize=\small' for more characters per line
\usepackage{framed}
\definecolor{shadecolor}{RGB}{241,243,245}
\newenvironment{Shaded}{\begin{snugshade}}{\end{snugshade}}
\newcommand{\AlertTok}[1]{\textcolor[rgb]{0.68,0.00,0.00}{#1}}
\newcommand{\AnnotationTok}[1]{\textcolor[rgb]{0.37,0.37,0.37}{#1}}
\newcommand{\AttributeTok}[1]{\textcolor[rgb]{0.40,0.45,0.13}{#1}}
\newcommand{\BaseNTok}[1]{\textcolor[rgb]{0.68,0.00,0.00}{#1}}
\newcommand{\BuiltInTok}[1]{\textcolor[rgb]{0.00,0.23,0.31}{#1}}
\newcommand{\CharTok}[1]{\textcolor[rgb]{0.13,0.47,0.30}{#1}}
\newcommand{\CommentTok}[1]{\textcolor[rgb]{0.37,0.37,0.37}{#1}}
\newcommand{\CommentVarTok}[1]{\textcolor[rgb]{0.37,0.37,0.37}{\textit{#1}}}
\newcommand{\ConstantTok}[1]{\textcolor[rgb]{0.56,0.35,0.01}{#1}}
\newcommand{\ControlFlowTok}[1]{\textcolor[rgb]{0.00,0.23,0.31}{\textbf{#1}}}
\newcommand{\DataTypeTok}[1]{\textcolor[rgb]{0.68,0.00,0.00}{#1}}
\newcommand{\DecValTok}[1]{\textcolor[rgb]{0.68,0.00,0.00}{#1}}
\newcommand{\DocumentationTok}[1]{\textcolor[rgb]{0.37,0.37,0.37}{\textit{#1}}}
\newcommand{\ErrorTok}[1]{\textcolor[rgb]{0.68,0.00,0.00}{#1}}
\newcommand{\ExtensionTok}[1]{\textcolor[rgb]{0.00,0.23,0.31}{#1}}
\newcommand{\FloatTok}[1]{\textcolor[rgb]{0.68,0.00,0.00}{#1}}
\newcommand{\FunctionTok}[1]{\textcolor[rgb]{0.28,0.35,0.67}{#1}}
\newcommand{\ImportTok}[1]{\textcolor[rgb]{0.00,0.46,0.62}{#1}}
\newcommand{\InformationTok}[1]{\textcolor[rgb]{0.37,0.37,0.37}{#1}}
\newcommand{\KeywordTok}[1]{\textcolor[rgb]{0.00,0.23,0.31}{\textbf{#1}}}
\newcommand{\NormalTok}[1]{\textcolor[rgb]{0.00,0.23,0.31}{#1}}
\newcommand{\OperatorTok}[1]{\textcolor[rgb]{0.37,0.37,0.37}{#1}}
\newcommand{\OtherTok}[1]{\textcolor[rgb]{0.00,0.23,0.31}{#1}}
\newcommand{\PreprocessorTok}[1]{\textcolor[rgb]{0.68,0.00,0.00}{#1}}
\newcommand{\RegionMarkerTok}[1]{\textcolor[rgb]{0.00,0.23,0.31}{#1}}
\newcommand{\SpecialCharTok}[1]{\textcolor[rgb]{0.37,0.37,0.37}{#1}}
\newcommand{\SpecialStringTok}[1]{\textcolor[rgb]{0.13,0.47,0.30}{#1}}
\newcommand{\StringTok}[1]{\textcolor[rgb]{0.13,0.47,0.30}{#1}}
\newcommand{\VariableTok}[1]{\textcolor[rgb]{0.07,0.07,0.07}{#1}}
\newcommand{\VerbatimStringTok}[1]{\textcolor[rgb]{0.13,0.47,0.30}{#1}}
\newcommand{\WarningTok}[1]{\textcolor[rgb]{0.37,0.37,0.37}{\textit{#1}}}

\providecommand{\tightlist}{%
  \setlength{\itemsep}{0pt}\setlength{\parskip}{0pt}}\usepackage{longtable,booktabs,array}
\usepackage{calc} % for calculating minipage widths
% Correct order of tables after \paragraph or \subparagraph
\usepackage{etoolbox}
\makeatletter
\patchcmd\longtable{\par}{\if@noskipsec\mbox{}\fi\par}{}{}
\makeatother
% Allow footnotes in longtable head/foot
\IfFileExists{footnotehyper.sty}{\usepackage{footnotehyper}}{\usepackage{footnote}}
\makesavenoteenv{longtable}
\usepackage{graphicx}
\makeatletter
\def\maxwidth{\ifdim\Gin@nat@width>\linewidth\linewidth\else\Gin@nat@width\fi}
\def\maxheight{\ifdim\Gin@nat@height>\textheight\textheight\else\Gin@nat@height\fi}
\makeatother
% Scale images if necessary, so that they will not overflow the page
% margins by default, and it is still possible to overwrite the defaults
% using explicit options in \includegraphics[width, height, ...]{}
\setkeys{Gin}{width=\maxwidth,height=\maxheight,keepaspectratio}
% Set default figure placement to htbp
\makeatletter
\def\fps@figure{htbp}
\makeatother

\usepackage{booktabs}
\usepackage{longtable}
\usepackage{array}
\usepackage{multirow}
\usepackage{float}
\usepackage{graphicx}
\makeatletter
\@ifpackageloaded{caption}{}{\usepackage{caption}}
\AtBeginDocument{%
\ifdefined\contentsname
  \renewcommand*\contentsname{Table of contents}
\else
  \newcommand\contentsname{Table of contents}
\fi
\ifdefined\listfigurename
  \renewcommand*\listfigurename{List of Figures}
\else
  \newcommand\listfigurename{List of Figures}
\fi
\ifdefined\listtablename
  \renewcommand*\listtablename{List of Tables}
\else
  \newcommand\listtablename{List of Tables}
\fi
\ifdefined\figurename
  \renewcommand*\figurename{Figure}
\else
  \newcommand\figurename{Figure}
\fi
\ifdefined\tablename
  \renewcommand*\tablename{Table}
\else
  \newcommand\tablename{Table}
\fi
}
\@ifpackageloaded{float}{}{\usepackage{float}}
\floatstyle{ruled}
\@ifundefined{c@chapter}{\newfloat{codelisting}{h}{lop}}{\newfloat{codelisting}{h}{lop}[chapter]}
\floatname{codelisting}{Listing}
\newcommand*\listoflistings{\listof{codelisting}{List of Listings}}
\makeatother
\makeatletter
\makeatother
\makeatletter
\@ifpackageloaded{caption}{}{\usepackage{caption}}
\@ifpackageloaded{subcaption}{}{\usepackage{subcaption}}
\makeatother
\ifLuaTeX
  \usepackage{selnolig}  % disable illegal ligatures
\fi
\usepackage{bookmark}

\IfFileExists{xurl.sty}{\usepackage{xurl}}{} % add URL line breaks if available
\urlstyle{same} % disable monospaced font for URLs
\hypersetup{
  colorlinks=true,
  linkcolor={blue},
  filecolor={Maroon},
  citecolor={Blue},
  urlcolor={Blue},
  pdfcreator={LaTeX via pandoc}}

\author{}
\date{}

\begin{document}

\section{Academic Paper Template - Setup
Guide}\label{academic-paper-template---setup-guide}

\begin{quote}
\textbf{Note:} This file documents the template setup and installation.
For research-specific information about a project using this template,
see \texttt{README.md}.
\end{quote}

A focused, user-friendly template for reproducible research using R with
Quarto. This template provides a clean starting point for academic
papers with strong reproducibility guarantees through R version pinning,
dependency locking, and automated workflows.

\subsection{Features}\label{features}

\begin{itemize}
\tightlist
\item
  \textbf{Simple \& Focused}: R-only template (no multi-language
  complexity)
\item
  \textbf{Reproducible}: R version pinning (\texttt{.Rversion}) +
  dependency locking (\texttt{renv})
\item
  \textbf{Well-documented}: Includes reproducibility guide and data
  documentation templates
\item
  \textbf{CI/CD Ready}: Automated rendering with package caching (5-10x
  speedup)
\item
  \textbf{Optional Extras}: Pre-commit hooks, Docker support, spell
  checking
\item
  \textbf{Generic Template}: \textasciitilde100 line starter (vs 435+
  line PLOS example)
\item
  \textbf{sessionInfo() Included}: Automatic computational environment
  documentation
\end{itemize}

\subsection{Directory Structure}\label{directory-structure}

\begin{Shaded}
\begin{Highlighting}[]
\NormalTok{paper{-}template/}
\NormalTok{├── .github/workflows/    \# CI/CD pipelines}
\NormalTok{├── data/}
\NormalTok{│   ├── raw/             \# Original, immutable data}
\NormalTok{│   └── processed/       \# Cleaned and processed data (gitignored)}
\NormalTok{├── scripts/}
\NormalTok{│   ├── R/               \# R analysis scripts}
\NormalTok{│   └── julia/           \# Julia analysis scripts}
\NormalTok{├── output/}
\NormalTok{│   ├── figures/         \# Generated figures (gitignored)}
\NormalTok{│   └── tables/          \# Generated tables (gitignored)}
\NormalTok{├── paper/}
\NormalTok{│   ├── index.qmd        \# Main paper document}
\NormalTok{│   ├── index.pdf        \# Rendered PDF (gitignored)}
\NormalTok{│   ├── references.bib   \# Bibliography}
\NormalTok{│   └── .wordlist.txt    \# Custom spelling dictionary}
\NormalTok{├── \_quarto.yml          \# Quarto configuration}
\NormalTok{├── Project.toml         \# Julia project dependencies}
\NormalTok{├── renv.lock            \# R dependencies lockfile}
\NormalTok{├── Makefile             \# Build automation}
\NormalTok{├── CLAUDE.md            \# AI assistant guidance}
\NormalTok{└── README.md            \# This file}
\end{Highlighting}
\end{Shaded}

\subsection{Do You Need This Template?}\label{do-you-need-this-template}

\subsubsection{✅ Use This Template If:}\label{use-this-template-if}

\begin{itemize}
\tightlist
\item
  Writing an academic paper with R code/analysis
\item
  Need version control and reproducibility
\item
  Want automated PDF rendering
\item
  Working with collaborators
\end{itemize}

\subsubsection{❌ Don't Use This If:}\label{dont-use-this-if}

\begin{itemize}
\tightlist
\item
  Simple report (use basic Quarto project)
\item
  No R code needed (use LaTeX or Markdown directly)
\item
  Just exploring R (too much infrastructure)
\end{itemize}

\subsubsection{Alternatives:}\label{alternatives}

\begin{itemize}
\tightlist
\item
  \textbf{Journal-specific formats}:
  \href{https://github.com/quarto-journals/}{quarto-journals}
\item
  \textbf{R package for articles}:
  \href{https://github.com/rstudio/rticles}{rticles}
\item
  \textbf{Simple Quarto project}:
  \texttt{quarto\ create\ project\ default\ my-paper}
\end{itemize}

\subsection{Prerequisites}\label{prerequisites}

\subsubsection{Required (Core
Functionality)}\label{required-core-functionality}

\begin{itemize}
\tightlist
\item
  \href{https://www.r-project.org/}{R} 4.5.1 (or version in
  \texttt{.Rversion})
\item
  \href{https://quarto.org/}{Quarto} (latest version)
\item
  \href{https://www.latex-project.org/}{LaTeX} (TeXLive or MacTeX)
\end{itemize}

\subsubsection{Optional (Enhanced
Features)}\label{optional-enhanced-features}

\begin{itemize}
\tightlist
\item
  \href{https://www.gnu.org/software/make/}{GNU Make} - build automation
\item
  \href{https://www.docker.com/}{Docker} - containerized reproducibility
\item
  \href{https://pre-commit.com/}{pre-commit} - code quality hooks (copy
  from \texttt{.example})
\item
  \href{http://aspell.net/}{aspell} - spell checking (CI only)
\end{itemize}

\subsubsection{Installation Commands}\label{installation-commands}

\textbf{macOS (using Homebrew):}

\begin{Shaded}
\begin{Highlighting}[]
\ExtensionTok{brew}\NormalTok{ install r quarto}
\ExtensionTok{brew}\NormalTok{ install }\AttributeTok{{-}{-}cask}\NormalTok{ mactex  }\CommentTok{\# LaTeX distribution}
\end{Highlighting}
\end{Shaded}

\textbf{Ubuntu/Debian:}

\begin{Shaded}
\begin{Highlighting}[]
\FunctionTok{sudo}\NormalTok{ apt{-}get update}
\FunctionTok{sudo}\NormalTok{ apt{-}get install r{-}base quarto{-}cli texlive{-}full}
\end{Highlighting}
\end{Shaded}

\subsection{Getting Started}\label{getting-started}

\subsubsection{1. Clone or Use This
Template}\label{clone-or-use-this-template}

Click ``Use this template'' on GitHub or clone directly:

\begin{Shaded}
\begin{Highlighting}[]
\FunctionTok{git}\NormalTok{ clone https://github.com/yourusername/paper{-}template.git my{-}research{-}paper}
\BuiltInTok{cd}\NormalTok{ my{-}research{-}paper}
\end{Highlighting}
\end{Shaded}

\subsubsection{2. Install R Dependencies}\label{install-r-dependencies}

\begin{Shaded}
\begin{Highlighting}[]
\ExtensionTok{Rscript} \AttributeTok{{-}e} \StringTok{"install.packages(\textquotesingle{}renv\textquotesingle{})"}
\ExtensionTok{Rscript} \AttributeTok{{-}e} \StringTok{"renv::restore()"}
\end{Highlighting}
\end{Shaded}

\subsubsection{3. Start Writing}\label{start-writing}

Edit \texttt{paper/index.qmd} to write your paper. The template includes
example sections, code chunks, and citations to guide you.

\subsection{Quick Start (5 Minutes)}\label{quick-start-5-minutes}

For the impatient:

\begin{Shaded}
\begin{Highlighting}[]
\CommentTok{\# Clone and setup}
\FunctionTok{git}\NormalTok{ clone https://github.com/yourusername/paper{-}template.git my{-}paper}
\BuiltInTok{cd}\NormalTok{ my{-}paper}

\CommentTok{\# Install R packages}
\ExtensionTok{Rscript} \AttributeTok{{-}e} \StringTok{"install.packages(\textquotesingle{}renv\textquotesingle{}); renv::restore()"}

\CommentTok{\# Render paper}
\ExtensionTok{quarto}\NormalTok{ render paper/index.qmd}

\CommentTok{\# View output}
\ExtensionTok{open}\NormalTok{ paper/index.pdf  }\CommentTok{\# macOS}
\CommentTok{\# or xdg{-}open paper/index.pdf  \# Linux}
\end{Highlighting}
\end{Shaded}

Done! Your PDF is in \texttt{paper/index.pdf}.

\subsection{Documentation}\label{documentation}

This repository includes \texttt{CLAUDE.md}, which provides
comprehensive guidance for AI assistants (like Claude Code) working with
this codebase. It contains:

\begin{itemize}
\tightlist
\item
  High-level architecture and data flow
\item
  Common development commands
\item
  Configuration details and conventions
\item
  CI/CD workflow patterns
\item
  Important non-obvious implementation details
\end{itemize}

Human developers may also find this useful for understanding the project
structure and workflows.

\subsection{Usage}\label{usage}

\subsubsection{Building the Paper}\label{building-the-paper}

Render the paper to PDF:

\begin{Shaded}
\begin{Highlighting}[]
\ExtensionTok{quarto}\NormalTok{ render paper/index.qmd}
\end{Highlighting}
\end{Shaded}

The rendered PDF will be \texttt{paper/index.pdf} (same directory as the
source file).

\subsubsection{Running Analyses}\label{running-analyses}

\begin{enumerate}
\def\labelenumi{\arabic{enumi}.}
\tightlist
\item
  Place raw data in \texttt{data/raw/}
\item
  Create analysis scripts in \texttt{scripts/R/} or
  \texttt{scripts/julia/}
\item
  Save processed data to \texttt{data/processed/}
\item
  Save figures to \texttt{output/figures/}
\item
  Save tables to \texttt{output/tables/}
\end{enumerate}

\subsubsection{Using R and Julia Code in
Quarto}\label{using-r-and-julia-code-in-quarto}

\textbf{R code chunk:}

\begin{Shaded}
\begin{Highlighting}[]
\CommentTok{\#| label: fig{-}example}
\CommentTok{\#| fig{-}cap: "Example figure"}

\FunctionTok{library}\NormalTok{(ggplot2)}
\FunctionTok{ggplot}\NormalTok{(data, }\FunctionTok{aes}\NormalTok{(x, y)) }\SpecialCharTok{+} \FunctionTok{geom\_point}\NormalTok{()}
\end{Highlighting}
\end{Shaded}

\textbf{Julia code chunk:}

\begin{Shaded}
\begin{Highlighting}[]
\CommentTok{\#| label: fig{-}example{-}julia}
\CommentTok{\#| fig{-}cap: "Julia figure"}

\ImportTok{using} \BuiltInTok{Plots}
\FunctionTok{plot}\NormalTok{(x, y)}
\end{Highlighting}
\end{Shaded}

\subsubsection{Managing Dependencies}\label{managing-dependencies}

\textbf{R dependencies (using renv):}

\begin{Shaded}
\begin{Highlighting}[]
\CommentTok{\# Install a new package}
\FunctionTok{install.packages}\NormalTok{(}\StringTok{"package\_name"}\NormalTok{)}

\CommentTok{\# Update renv.lock}
\NormalTok{renv}\SpecialCharTok{::}\FunctionTok{snapshot}\NormalTok{()}

\CommentTok{\# Restore packages}
\NormalTok{renv}\SpecialCharTok{::}\FunctionTok{restore}\NormalTok{()}
\end{Highlighting}
\end{Shaded}

\subsubsection{Using Make (Optional)}\label{using-make-optional}

The template includes a basic Makefile with examples. Customize it for
your workflow:

\begin{Shaded}
\begin{Highlighting}[]
\FunctionTok{make}        \CommentTok{\# Show available targets}
\FunctionTok{make}\NormalTok{ help   }\CommentTok{\# Show detailed help}
\FunctionTok{make}\NormalTok{ clean  }\CommentTok{\# Remove generated files (.quarto/, paper/*.pdf, paper/*.tex)}
\end{Highlighting}
\end{Shaded}

See the Makefile for example recipes you can uncomment or customize.

\subsection{Bibliography Management}\label{bibliography-management}

Add references to \texttt{paper/references.bib} in BibTeX format:

\begin{Shaded}
\begin{Highlighting}[]
\VariableTok{@article}\NormalTok{\{}\OtherTok{author2024}\NormalTok{,}
  \DataTypeTok{title}\NormalTok{ = \{Article Title\},}
  \DataTypeTok{author}\NormalTok{ = \{Author, Name\},}
  \DataTypeTok{year}\NormalTok{ = \{2024\},}
  \DataTypeTok{journal}\NormalTok{ = \{Journal Name\},}
  \DataTypeTok{volume}\NormalTok{ = \{1\},}
  \DataTypeTok{pages}\NormalTok{ = \{1{-}{-}10\}}
\NormalTok{\}}
\end{Highlighting}
\end{Shaded}

Cite in text: \texttt{{[}@author2024{]}} or \texttt{@author2024}

\subsection{Customization}\label{customization}

\subsubsection{Changing Citation Style}\label{changing-citation-style}

Edit \texttt{\_quarto.yml} to use a different CSL style:

\begin{Shaded}
\begin{Highlighting}[]
\FunctionTok{csl}\KeywordTok{:}\AttributeTok{ https://www.zotero.org/styles/nature}
\end{Highlighting}
\end{Shaded}

Browse styles at \href{https://www.zotero.org/styles}{Zotero Style
Repository}

\subsubsection{Changing PDF Formatting}\label{changing-pdf-formatting}

Modify the \texttt{format.pdf} section in \texttt{\_quarto.yml}:

\begin{Shaded}
\begin{Highlighting}[]
\FunctionTok{format}\KeywordTok{:}
\AttributeTok{  }\FunctionTok{pdf}\KeywordTok{:}
\AttributeTok{    }\FunctionTok{documentclass}\KeywordTok{:}\AttributeTok{ article}
\AttributeTok{    }\FunctionTok{fontsize}\KeywordTok{:}\AttributeTok{ 12pt}
\AttributeTok{    }\FunctionTok{geometry}\KeywordTok{:}
\AttributeTok{      }\KeywordTok{{-}}\AttributeTok{ margin=1.5in}
\end{Highlighting}
\end{Shaded}

\subsubsection{Adding Custom LaTeX
Packages}\label{adding-custom-latex-packages}

Edit the \texttt{include-in-header} section in \texttt{\_quarto.yml}.

\subsection{CI/CD}\label{cicd}

This template includes GitHub Actions workflows:

\subsubsection{render.yml}\label{render.yml}

Automatically renders the paper when you push changes to:

\begin{itemize}
\tightlist
\item
  Paper content (\texttt{paper/})
\item
  Analysis scripts (\texttt{scripts/})
\item
  Data files (\texttt{data/})
\item
  Configuration (\texttt{\_quarto.yml})
\item
  Dependencies (\texttt{renv.lock}, \texttt{.Rprofile})
\end{itemize}

\textbf{Performance optimizations:}

\begin{itemize}
\tightlist
\item
  R package caching via \texttt{r-lib/actions/setup-renv@v2} (5-10x
  speedup after first run)
\item
  R version pinning (4.5.1) for reproducibility
\item
  Complete LaTeX support including \texttt{texlive-bibtex-extra} for
  bibliographies
\end{itemize}

Rendered PDFs are available as artifacts in GitHub Actions. The workflow
typically takes 2-3 minutes after caching is established (vs.~10-20
minutes without caching).

\subsubsection{checks.yml}\label{checks.yml}

Runs spell checking with aspell using the custom dictionary in
\texttt{paper/.wordlist.txt}. Generates warnings but doesn't fail the
build, with errors saved as artifacts.

\subsection{Docker Support (Optional)}\label{docker-support-optional}

For extreme reproducibility, use Docker:

\begin{Shaded}
\begin{Highlighting}[]
\CommentTok{\# Build image}
\ExtensionTok{docker}\NormalTok{ build }\AttributeTok{{-}t}\NormalTok{ my{-}paper .}

\CommentTok{\# Render paper in container}
\ExtensionTok{docker}\NormalTok{ run }\AttributeTok{{-}{-}rm} \AttributeTok{{-}v} \VariableTok{$(}\BuiltInTok{pwd}\VariableTok{)}\NormalTok{:/project my{-}paper}

\CommentTok{\# Or run interactively}
\ExtensionTok{docker}\NormalTok{ run }\AttributeTok{{-}{-}rm} \AttributeTok{{-}it} \AttributeTok{{-}v} \VariableTok{$(}\BuiltInTok{pwd}\VariableTok{)}\NormalTok{:/project my{-}paper bash}
\end{Highlighting}
\end{Shaded}

The Docker image guarantees:

\begin{itemize}
\tightlist
\item
  Exact R version (4.5.1)
\item
  Exact system dependencies
\item
  Consistent LaTeX environment
\item
  Same results across all systems
\end{itemize}

See \texttt{Dockerfile} for details.

\subsection{Reproducibility}\label{reproducibility}

This template emphasizes computational reproducibility:

\begin{itemize}
\tightlist
\item
  \textbf{R Version Pinning}: \texttt{.Rversion} file + automatic
  checking in \texttt{.Rprofile}
\item
  \textbf{Package Locking}: \texttt{renv.lock} with exact versions
\item
  \textbf{sessionInfo()}: Automatically included in paper appendix
\item
  \textbf{Data Documentation}: Template in \texttt{data/README.md}
\item
  \textbf{Best Practices Guide}: See \texttt{REPRODUCIBILITY.md} for
  detailed guidance
\end{itemize}

\textbf{Quick reproducibility check:}

\begin{Shaded}
\begin{Highlighting}[]
\CommentTok{\# Verify R version matches}
\FunctionTok{cat}\NormalTok{ .Rversion}

\CommentTok{\# Check package synchronization}
\ExtensionTok{Rscript} \AttributeTok{{-}e} \StringTok{"renv::status()"}

\CommentTok{\# Test clean render}
\FunctionTok{make}\NormalTok{ clean }\KeywordTok{\&\&} \ExtensionTok{quarto}\NormalTok{ render paper/index.qmd}
\end{Highlighting}
\end{Shaded}

\subsection{Contributing}\label{contributing}

When using this template:

\begin{enumerate}
\def\labelenumi{\arabic{enumi}.}
\tightlist
\item
  Create a new branch for your work
\item
  Make your changes
\item
  Pre-commit hooks will automatically format your code
\item
  Push and create a pull request
\item
  CI/CD will render the paper and run checks
\end{enumerate}

\subsection{License}\label{license}

This template is licensed under the MIT License - see the \url{LICENSE}
file for details.

\subsection{Troubleshooting}\label{troubleshooting}

\subsubsection{Quarto not rendering}\label{quarto-not-rendering}

Ensure all dependencies are installed:

\begin{Shaded}
\begin{Highlighting}[]
\ExtensionTok{quarto}\NormalTok{ check install}
\end{Highlighting}
\end{Shaded}

\subsubsection{R package installation
fails}\label{r-package-installation-fails}

Try updating renv:

\begin{Shaded}
\begin{Highlighting}[]
\ExtensionTok{Rscript} \AttributeTok{{-}e} \StringTok{"install.packages(\textquotesingle{}renv\textquotesingle{}, repos=\textquotesingle{}https://cloud.r{-}project.org\textquotesingle{})"}
\ExtensionTok{Rscript} \AttributeTok{{-}e} \StringTok{"renv::restore()"}
\end{Highlighting}
\end{Shaded}

\subsubsection{Julia packages not found}\label{julia-packages-not-found}

Instantiate the Julia environment:

\begin{Shaded}
\begin{Highlighting}[]
\ExtensionTok{julia} \AttributeTok{{-}{-}project}\OperatorTok{=}\NormalTok{. }\AttributeTok{{-}e} \StringTok{\textquotesingle{}using Pkg; Pkg.instantiate()\textquotesingle{}}
\end{Highlighting}
\end{Shaded}

\subsubsection{LaTeX errors}\label{latex-errors}

Install the full TeXLive distribution:

\begin{itemize}
\tightlist
\item
  macOS: \texttt{brew\ install\ -\/-cask\ mactex}
\item
  Ubuntu: \texttt{sudo\ apt-get\ install\ texlive-full}
\end{itemize}

\subsection{Resources}\label{resources}

\begin{itemize}
\tightlist
\item
  \href{https://quarto.org/docs/guide/}{Quarto Documentation}
\item
  \href{https://r4ds.had.co.nz/}{R for Data Science}
\item
  \href{https://docs.julialang.org/}{Julia Documentation}
\item
  \href{http://www.bibtex.org/Using/}{BibTeX Guide}
\item
  \href{https://pre-commit.com/}{Pre-commit Hooks}
\end{itemize}

\subsection{Citation}\label{citation}

If you use this template, please consider citing it:

\begin{Shaded}
\begin{Highlighting}[]
\VariableTok{@misc}\NormalTok{\{}\OtherTok{papertemplate2025}\NormalTok{,}
  \DataTypeTok{title}\NormalTok{ = \{Academic Paper Template\},}
  \DataTypeTok{author}\NormalTok{ = \{Azam, James\},}
  \DataTypeTok{year}\NormalTok{ = \{2025\},}
  \DataTypeTok{url}\NormalTok{ = \{https://github.com/yourusername/paper{-}template\}}
\NormalTok{\}}
\end{Highlighting}
\end{Shaded}




\end{document}
